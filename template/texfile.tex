%!TEX program=xelatex
%!Mode::"TeX:UTF-8"
%report bugs to liyu@tjufe.edu.cn
\documentclass[12pt,a4paper]{article}
\usepackage{amsmath,amsthm}
\usepackage{hyperref}
\usepackage{color}
\usepackage[titletoc]{appendix}
\usepackage{fontspec}
\setmainfont{Times New Roman}   %西文默认衬线字体
\setsansfont{Arial}             %西文默认无衬线字体
\setmonofont{Courier New}       %西文默认等宽字体
\usepackage{xeCJK}
\setCJKmainfont[BoldFont={SimHei},
 ItalicFont={KaiTi}]{FangSong}  %中文衬衣字体
\setCJKsansfont{SimHei}         %中文无衬衣字体
\setCJKmonofont{KaiTi}          %中文等宽字体
\bibliographystyle{apalike}
\synctex=1
\newtheorem{theorem}{Theorem}[section]
\begin{document}
\title{\bf Hello TeX, 你好TeX}
\author{
Hello author\footnote{{\tt author@email}},
你好作者\footnote{{\tt 电子邮件}}
}
\date{\today}
\maketitle
\begin{abstract}
Hello abstract, 你好摘要
\vskip0.3cm {\bf Keywords.} Hello Keywords, 你好关键字.
\end{abstract}

\section{Hello section, 你好节}
{\color{red}Hello section}, 你好节
\cite{Babuska1989Finite, Brenner2008Mathematical},
\cite{2004有限元方法的数学基础}
\subsection{Hello subsection, 你好子节}
Hello subsection, 你好子节
\begin{theorem}
\begin{equation*}
\int_{a}^{b} f(x) \,dx
\end{equation*}
\end{theorem}
\begin{proof}
\begin{align*}
&\frac{\partial^2 f}{\partial^2 x}+\frac{\partial^2 f}{\partial^2 y} = \sin(x+y), \\
&\frac{\partial^2 g}{\partial^2 x}+\frac{\partial^2 g}{\partial^2 y} = \cos(x+y).
\end{align*}

\begin{equation*}
f(x)=\begin{cases}
\infty,  & \mbox{if } x=\alpha+\gamma,  \\
\lim\limits_{y\rightarrow 0}\sqrt{y},  & \mbox{otherwise}.
\end{cases}
\end{equation*}
\end{proof}



\bibliography{bibfile}

\begin{appendices}
\section{Hello appendices, 你好附录}
Hello appendices, 你好附录
\end{appendices}
\end{document}
